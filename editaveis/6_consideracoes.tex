\chapter{CONSIDERAÇÕES FINAIS}
\label{chapter:final}

As contribuições deste trabalho propiciaram ao InterSCity a possibilidade de
atuação em cenários mais abrangentes, por meio de um novo serviço
para processamento de seus dados. Juntamente com o novo serviço, desenvolvemos
uma aplicação que abstrai as ferramentas de \textit{Big Data} utilizadas pelo
projeto, e que fornece meios para que outras aplicações de cidades inteligentes
requisitem tarefas para serem processadas pela plataforma. Com a incorporação
dos resultados que desenvolvemos o InterSCity passa então a poder atuar em
operações com maior massa de dados e a fornecer processamento de dados para
terceiros através da plataforma.

Nesta segunda etapa do trabalho, que começou em maio/2017, focamos em
amadurecer o Shock e em definir casos de uso que pudessem ser resolvidos com
a sua utilização - o que nos levou ao desenvolvimento do Forensic, que apresenta
uma interface mais amigável ao usuário final. Com relação ao planejado, embora
tenhamos dado foco em amadurecer o Shock e a solucionar cenários de uso, ainda
assim implementamos parte das atividades planejadas relacionadas a evolução e
funcionalidades, por serem importantes em cenários de uso mais reais. Foram
elas:

\begin{itemize}
    \item \textbf{Documentar a API de serviços:} Conseguimos dar maior ênfase a
        documentação da API do serviço desenvolvido, que podem ser conferidas
        num servidor específico para isso, desenvolvido em Sphinx;
    \item \textbf{Utilização dos \textit{data frames}:} A utilização dos
        \textit{data frames} possibilitou o uso de síntaxe SQL, de menor
        complexidade que a manipulação de \textit{RDDs};
    \item \textbf{Múltiplos \textit{streams}:} Mudamos a API de \textit{stream}
        para fazer uso dos \textit{streams} estruturados, que possibilitam o
        uso de \textit{streams} múltiplos, o que possibilita que uma instância
        do serviço de processamento possa ser utilizado por mais de um usuário.
\end{itemize}

Além das novas funcionalidades, a partir do desenvolvimento deste TCC, foram
feitas outras contribuições, como (i) a criação de um novo \textit{script} de
coleta de dados, que consome a API de estações de bicicleta; e (ii) conserto
de um \textit{script} que ingere dados sobre a qualidade do ar.
