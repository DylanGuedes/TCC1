\chapter{CONSIDERAÇÕES FINAIS}
\label{chapter:final}

As contribuições deste trabalho propiciaram ao InterSCity a possibilidade de
atuação em cenários mais abrangentes, por meio de um novo serviço
para processamento de seus dados. Juntamente com o novo serviço, desenvolvemos
o Shock, uma aplicação que abstrai e gerencia o uso de ferramentas de
\textit{Big Data}, e o Forensic, que visa demonstrar a utilização do Shock.
O Forensic é necessário pois sua interface aproxima os usuários finais, que
passam a poder gerenciar processamento de fluxos de dados complexos sem
conhecimento profundo das tecnologias.
Com a incorporação dos resultados que desenvolvemos, o InterSCity passa a poder
atuar em operações com maior massa de dados e a fornecer processamento de dados
para terceiros através da plataforma.

Nesta segunda etapa do trabalho, que começou em maio/2017, focamos em
amadurecer o Shock e em definir casos de uso que pudessem ser resolvidos com
a sua utilização - o que nos levou ao desenvolvimento do Forensic, que apresenta
uma interface mais amigável ao usuário final. Com relação ao planejado, embora
tenhamos dado foco em amadurecer o Shock e a solucionar cenários de uso, ainda
assim implementamos parte das atividades planejadas relacionadas a evolução e
funcionalidades, por serem importantes em cenários de uso mais reais. Foram
elas:

\begin{itemize}
    \item \textbf{Documentar a API de serviços:} Conseguimos dar maior ênfase a
        documentação da API do serviço desenvolvido, que podem ser conferidas
        num servidor específico para isso\footnote{\url{http://shock.readthedocs.io/en/latest/index.html}},
        desenvolvido em Sphinx;
    \item \textbf{Utilização dos \textit{data frames}:} A utilização dos
        \textit{data frames} possibilitou o uso de síntaxe SQL, de menor
        complexidade que a manipulação de \textit{RDDs};
    \item \textbf{Múltiplos fluxos de dados:} Mudamos a API de fluxos de dados
        para que fosse feito o uso de fluxo estruturado, que
        possibilitam o uso de fluxos múltiplos, adicionando ao
        InterSCity a possibilidade de, com apenas uma instância do ecossistema
        da plataforma, servir processamento de dados para vários usuários.
\end{itemize}

Além das novas funcionalidades, a partir do desenvolvimento deste TCC, foram
feitas outras contribuições, como (i) a criação de um novo \textit{script} de
coleta de dados, que consome a API de estações de bicicleta; e (ii) melhoria 
de um \textit{script} que ingere dados sobre a qualidade do ar.

As limitações das aplicações desenvolvidas se devem principalmente aos cenários
alternativos, como edição e visualização de \textit{log} dos fluxos,
que não foram desenvolvidos. Atualmente não é possível, por exemplo, a edição de
um fluxo que já está em estado de processamento, a abortagem do
processamento de um fluxo ou a visualização do \textit{log} do
\textit{status} de um fluxo pelo Forensic. Com base nessas limitações,
definimos como os próximos passos para o Shock e Forensic a adição de cenários
alternativos para os fluxos de processamento, um teste em cenário real de cidades
inteligentes e a incorporação do serviço de processamento ao núcleo do
InterSCity.
