\chapter[INTERSCITY]{INTERSCITY}
\label{chapter:interscity}

O InterSCity está licenciado sob MPLv2\footnote{\url{www.mozilla.org/en-US/MPL/2.0/}},
e foi construído com a utilização da arquitetura de microserviços -
MSA\footnote{\url{microservices.io/}}
\cite{delesposte2017}. Baseando-se no desenvolvimento colaborativo e na
utilização de tecnologias software livre de ponta, o projeto é desenvolvido
com a ajuda de diversos colaboradores, que, utilizando metodologias ágeis,
possibilitam a manutenção e evolução da plataforma ao longo do tempo \cite{delesposte2017}.

A maior parte dos microserviços\footnote{Os termos microserviços, módulos, e
componentes, serão utilizados intermitentemente, mas apresentando o mesmo
significado.} da plataforma foram escritos em Ruby on Rails\footnote{\url{rubyonrails.org/}},
seguindo padrões que priorizam extensibilidade e qualidade, e levando em conta
princípios\footnote{Os princípios seguidos pelo InterSCity são apresentados no
Apêndice \ref{appendix:principles}} que levaram a uma arquitetura madura,
robusta, e extensível. A partir de experimentos feitos\footnote{Informações
importantes sobre o experimento estão disponíveis do Apêndice \ref{appendix:performance}
}, é possível conferir quão promissor é o InterSCity. O projeto encontra-se
hospedado no Gitlab\footnote{\url{gitlab.com/smart-city-software-platform}},
onde é possível ter acesso ao código fonte, documentação, e um exemplo de
cliente que ilustra o uso da plataforma.

\section{ARQUITETURA}

O InterSCity apresenta uma arquitetura de microserviços distribuída, composta
pelos microserviços Resource Adaptor, Resource Viewer, Resource Catalog,
Data Collector, Resource Discovery, e Actuator Controller. Estes componentes
são desacoplados entre si, e cada um tem responsabilidades específicas e bem
definidas. A comunicação entre eles é feita via requisições REST e passagem de
mensagem. O modelo de comunicação via passagem de mensagem é importante em
contextos de concorrência pelo isolamento provido \cite{armstrong2003}, e é
feito utilizando o RabbitMQ\footnote{\url{www.rabbitmq.com/}}, através do
padrão de projeto PubSub\footnote{\url{xmpp.org/extensions/xep-0060.html}}.

\begin{figure}
  \centering
    \includegraphics[scale=0.5]{figuras/interscity-flow.png}
  \caption{Ciclo de vida de um recurso IoT no InterSCity.}
  \label{fig:interscity-lifecycle}
\end{figure}

A Figura \ref{fig:interscity-lifecycle} ilustra o ciclo de vida típico de um
recurso IoT na plataforma. Inicialmente, o recurso (i) faz um pedido de
registro na plataforma ao Resource Adaptor, que, (ii) cadastra o módulo no
microserviço Resource Catalog, (iii) e informa então ao recurso seu UUID
(identificador único), que será utilizado internamente deste passo em diante.
Após, a comunicação entre o Resource Adaptor e o dispositivo IoT terá
continuidade, mas, (iv) os dados terão como destino o módulo Data Collector,
que armazenará as informações. Por fim, (v) as informações contidas no
Data Collector serão enviadas para o Resource Viewer, apresentando então os
dados do recurso para o usuário final.

Como dito, os microserviços do InterSCity têm responsabilidades atômicas e bem
definidas, princípio chave para que a plataforma atinja o que propõe-se. O
microserviço \textbf{Resource Adaptor} é o grande responsável pela comunicação
entre os dispositivos IoT e a plataforma, funcionando como um mediador
durante as requisições \cite{delesposte2017}.

O \textbf{Data Collector} e o \textbf{Resource Catalog} tem papeis parecidos,
mas enquanto o primeiro gerencia e armazena dados históricos de medições dos
dispositivos, o segundo tem o papel de gerenciar e armazenar o registro dos
dispositivos na plataforma \cite{delesposte2017}.

O \textbf{Resource Viewer} e o \textbf{Resource Discovery}, por outro lado,
são similares por manipularem e utilizarem o Data Collector e o Resource
Catalog em sua execução. O Resource Viewer tem como objetivo apresentar ao
usuário final os dados dos recursos, enquanto o Resource Discovery provê formas
de busca por dispositivos disponíveis, possibilitando o uso de filtros
\cite{delesposte2017}. Por fim, o \textbf{Actuator Controller} provê serviços
para requisições nos recursos IoT atuadores registrados na plataforma
, armazenando os dados, possibilitando auditoria
\cite{delesposte2017}.

\section{GERÊNCIA DE CONFIGURAÇÃO E DEPENDÊNCIAS}

Outro aspecto que recebe atenção no desenvolvimento do InterSCity é a gerência
de configuração, que, atualmente, utiliza tecnologias que reduzem o esforço
desnecessário, promovem o isolamento entre a plataforma e o ambiente hospedeiro,
e aumentam a segurança no desenvolvimento. A gerência de configuração do
InterSCity é guiada por containêres do
Docker\footnote{\url{https://www.docker.com/}}, e cada microserviço e
dependência externa (como o RabbitMQ) são executados em
containêres separados. O desenvolvimento da plataforma também é
orientado pelo uso do Git\footnote{\url{https://git-scm.com/}}, de modo que a
configuração de um ambiente para execução do InterSCity tenha como
pré-requisitos somente estes dois projetos: Docker, e Git.

O InterSCity usa um esquema não tão comum quanto ao uso dos repositórios de
seus microserviços. Um repositório principal, chamado
\textit{dev-env}\footnote{\url{https://gitlab.com/smart-city-software-platform/dev-env}},
funciona como \textit{repositório mestre}, e os microserviços da plataforma são
submódulos pertencentes a este. Desta forma, os estados do projeto mestre são
ligados aos estados dos microserviços, garantindo compatibilidade entre estes.
Com essa abordagem, versões do repositório mestre garantem compatibilidade
entre os microserviços, dificultando cenários de incompatibilidade de API.

\section{PROPOSTA E METODOLOGIA}

O estado atual do InterSCity é estável, e, embora não conte com uma camada de
processamento de dados ideal, certo esforço culminou em uma camada de
processamento provisória, e que será utilizada como base para o desenvolvimento
da nova arquitetura. Houve ainda a troca de tecnologia de banco de dados, que
passou do Postgres (tecnologia SQL) para o MongoDB (tecnologia NoSQL). Essa
troca foi um passo importante na busca por uma maior elasticidade no volume de
dados, e facilitará o desenvolvimento da nova arquitetura. A equipe do
InterSCity nomeou a camada provisória de \textbf{Data Processor}, que,
atualmente, conta com uma configuração pronta para uso do Apache Spark, bem
como \textit{scripts} que ilustram situações de uso desta tecnologia, como
para consumir e processar dados de um dos exemplos disponíveis no
repositório.

A proposta de camada de processamento de dados, ao contrário da camada
provisória, deverá seguir um padrão de projeto para compor sua arquitetura, que
deve ser adequada para o contexto de cidades inteligentes, e compatível com a
arquitetura atual do InterSCity. Um levantamento das arquiteturas Big Data
candidatas deve ser feito, bem como um estudo a respeito das ferramentas que
as compõem. Este levantamento terá algumas restrições, como: somente projetos
software livre deverão ser levados em conta, e ferramentas que necessitem
grandes mudanças no ecossistema do InterSCity terão pouca prioridade.

É esperado então que seja desenvolvida uma nova arquitetura de processamento
de dados que atenda requisitos típicos de cidades inteligentes, e que possibilite
extensibilidade para trabalhos futuros. A nova arquitetura deve permitir,
por exemplo, que um \textit{pipeline de dados} possa ser utilizado, mesmo que
faça uso de uma grande massa de dados. Por fim, será fornecido um exemplo de
aplicação que utilize a arquitetura desenvolvida, como prova de conceito.
