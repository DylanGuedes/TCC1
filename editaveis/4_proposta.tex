\chapter[PROPOSTA]{PROPOSTA}

Como dito, o InterSCity já está maduro, mas falta a parte de processamento.
Neste capítulo eu vou definir as ferramentas e justificar as escolhas. E, ao fim,
levantarei o que será implementado de fato neste trabalho.

\section{ARQUITETURA}

A arquitetura definida para a camada de processamento de dados do InterSCity é a
Arquitetura Kappa. A Arquitetura Lambda embora já seja difundida em projetos
de excelência, é muito complexa para o contexto em questão - manter duas bases
de código, grandes e complexas, desvia de um dos focos do InterSCity, que é a
manutenibilidade. Dessa forma, a escolha pela Arquitetura Kappa parece ser a mais
adequada no contexto do InterSCity.

\section{TECNOLOGIAS}

As tecnologias definidas para a proposta foram feitas levando em conta as
seguintes prioridades:

\begin{enumerate}
    \item \textit{Software} livre;
    \item Tecnologias já usadas no InterSCity;
    \item Interoperabilidade com as tecnologias já usadas terão prioridade;
    \item Melhor manutenibilidade;
    \item Melhor escalabilidade e performance.
\end{enumerate}

Tecnologias de \textit{software} livre tiveram prioridade máxima pelo
ecossistema de Big Data ser composto de ferramentas livres de excelência, e por
ser uma das características do InterSCity. Tecnologias já usadas pelo
InterSCity também tiveram grande peso por não forçarem grandes mudanças ao
ecossistema já estabelecido, e por ter-se confiança nas escolhas feitas pelo
time do InterSCity. Os outros fatores de prioridade já são focos do InterSCity,
mas foi dada prioridade a manutenibilidade a performance, pelo contexto atual
do InterSCity não sofrer problemas de performance.

\subsection{Broker}

O \textit{broker} que será utilizado para transmissão dos dados é o
\textbf{RabbitMQ}. O principal critério utilizado foi o fato dele já ser usado
em larga escala pelo InterSCity, de modo que a troca de tecnologias não fosse
interessante, e o fato dele não comprometer nenhum critério. Contudo, têm-se
noção que o Kafka seria uma solução tão boa quanto, tendo ainda performance
superior ao RabbitMQ, e disponibilizando a funcionalidade de \textit{log} de
transações, que seria de grande utilidade na Arquitetura Kappa.

\subsection{Streaming}

A tecnologia de \textit{streaming} decidida é o \textbf{Apache Spark}. O Apache
Spark, como já levantado, tem alta performance, atende bem os requisitos, e tem
suporte para várias tecnologias, como Python, R, Scala e Java. Mas seu grande
diferencial é o fato de ter integrado multiplas formas de processamento
(\textit{batch} e \textit{streaming}), biblioteca para
\textit{machine learning} e \textit{clusterização}, entre outras utilidades,
como biblioteca de grafos. Por ser tão abrangente, o Spark seria a única tecnologia
necessária quanto ao processamento de dados, seguindo assim o princípio de
\textit{framework} único da Arquitetura Kappa. Além disso, uma transição para a
Arquitetura Lambda seria fácil caso fosse de interesse do InterSCity, pelo
Spark já abrangir as duas camadas de processamento.

\subsection{Banco de dados NoSQL}

A tecnologia de banco de dados NoSQL definida é o \textbf{MongoDB}. O principal
motivo é o fato dele já ser usado pelo InterSCity, e o fato dele não
comprometer nenhum critério. Contudo, têm-se noção que o Apache Cassandra
também seria uma ótima solução no contexto do InterSCity, pois já tem
\textit{handlers} para o Spark e para o RabbitMQ.

\section{IMPLEMENTAÇÃO}

% * Implementação inicial: levantar o spark, passar dados pelo rabbitmq e
% processa-los no spark
% * Qual case?
% * Usar a api da USP dos onibus, ver se algum onibus saiu fora da rota, ver
% se tá ocorrendo engarrafamento
