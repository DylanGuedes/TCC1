\begin{resumo}[Abstract]
 \begin{otherlanguage*}{english}

     InterSCity is a smart cities platform based on a microservices
     architecture that aims at supporting smart cities applications, offering a
     set of reusable, interoperable, and scalable services. However, InterSCity
     does not use suitable tools to process its data, an obstacle in scenarios
     of larger data set. This work aims to design and to implement a data
     processing service that allows InterSCity to handle larger data set, using
     mainly Big Data, a key technology for smart cities. With the adoption of
     state-of-the-art data processing architectures and new Big Data
     technologies, we expect InterSCity to be able to support more
     sophisticated applications for smart cities.

   \vspace{\onelineskip}
 
   \noindent 
   \textbf{Key-words}: Smart Cities, Big Data, Kappa Architecture.
 \end{otherlanguage*}
\end{resumo}
