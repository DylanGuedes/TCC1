\begin{resumo}[Abstract]
 \begin{otherlanguage*}{english}

     InterSCity is a smart cities platform based on a microservices
     architecture, and aims to support smart cities applications offering a
     set of reusable, interoperable, and scalable services. However, the usage
     of suitable tools for processing its data is not yet present, being an
     obstacle in scenarios of larger data set. This work aims to design and to
     iplement a data processing service that allows InterSCity to handle larger
     data set, using mainly Big Data, a key technology for smart cities. With
     the adoption of state-of-the-art data processing architectures and new
     Big Data technologies, we expect InterSCity to be able to support
     more sophisticated applications for smart cities.

   \vspace{\onelineskip}
 
   \noindent 
   \textbf{Key-words}: smart cities, big data, kappa architecture, lambda architecture
 \end{otherlanguage*}
\end{resumo}
