\chapter[INTRODUÇÃO]{INTRODUÇÃO}
\label{chapter:intro}

O termo \textbf{cidades inteligentes} recebe cada vez mais atenção, e trata-se
da utilização de tecnologias da informação e comunicação (TIC) para melhorar
setores como segurança, transporte, e saúde, aumentando a qualidade de vida
da população \cite{batty2012smart}. As cidades inteligentes ganham força por
atingirem soluções e mitigações concretas para problemas graves e recorrentes
das cidades atuais, como o mau uso de recursos, a burocracia, transporte de má
qualidade, e falta de segurança \cite{batty2012smart}.

Diversas iniciativas de cidades inteligentes ocorrem atualmente. Em Santander,
na Espanha, uma plataforma foi
desenvolvida\footnote{\url{www.smartsantander.eu/}}, e, utilizando-a,
diversos aplicativos foram criados, como para apresentar aos usuários lugares
livres para estacionar\footnote{\url{www.smartsantander.eu/wiki/index.php/Mitos/Mitos}},
ou visualização da poluição do ar da cidade \cite{santana2016software}. Em
Amsterdã, na Holanda, um projeto permite que os habitantes acompanhem em tempo
real o consumo de energia de suas casas \cite{kon2016}. Contudo, as plataformas
e soluções atuais, no geral, apresentam problemas. A abordagem utilizada nestas
soluções costuma ser muito específica e não padronizada, não se preocupando com
a interoperabilidade dentre os diversos projetos, e não promovendo
reaproveitamento das soluções existentes \cite{delesposte2017}.

Com a finalidade de ser uma plataforma que, em sua origem, se atente aos
problemas citados, surge o InterSCity, que busca promover interoperabilidade,
padronização, escalabilidade, e extensibilidade \cite{delesposte2017}. O
InterSCity está em desenvolvimento, e, embora já tenha uma arquitetura bem
definida, e esteja completamente funcional, ainda não conta com sua camada de
processamento de dados adequada para contextos de larga escala. Este trabalho
tem então como principal contribuição o planejamento, desenho, e implementação,
de uma camada de processamento de dados que forneça ao InterSCity o que é
necessário para que seja possível performar operações com grande massa de dados.
Isto será possível graças ao uso de tecnologias Big Data, que são tecnologias
chave para cidades inteligentes\cite{batty2012smart}.

Maiores detalhes sobre as características e o estado atual do InterSCity serão
apresentados no Capítulo \ref{chapter:interscity}, trazendo ainda a abordagem que será
seguida para definições a cerca da camada de processamento, e o que deverá ser
desenvolvido. O Capítulo \ref{chapter:data} trará um estudo sobre o estado da arte
em arquiteturas e tecnologias Big Data, apresentando uma análise a cerca das
ferramentas mais adequadas para o contexto da plataforma. O Capítulo
\ref{chapter:architecture} levantará as decisões e justificativas para compor
arquitetura, e, por fim, o Capítulo \ref{chapter:final} finaliza o trabalho,
trazendo as considerações iniciais e os próximos passos a serem tomados.
