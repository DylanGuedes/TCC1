\chapter[INTRODUÇÃO]{INTRODUÇÃO}

% \addcontentsline{toc}{chapter}{Introdução}

%TODO: falar sobre processamento de dados em cidades inteligentes

Baseado na ideia da utilização de tecnologias para melhorar setores como
segurança, transporte, e saúde, o termo cidades inteligentes fica cada vez
mais comum, recebendo assim cada vez mais atenção \cite{batty2012smart}.

Podendo ser definido como a utilização de Tecnologias da Informação e
Comunicação (TIC) para melhorar a vida dos cidadãos \cite{batty2012smart},
as cidades inteligentes ganham força por abrirem soluções e mitigações
concretas para problemas graves e recorrentes das cidades atuais. Problemas
como o mau uso de recursos, a burocracia, transporte de má qualidade,
e falta de segurança, podem ser resolvidos com soluções que utilizem TIC
\cite{batty2012smart}, daí sua relevância.

Diversas iniciativas de cidades inteligentes ocorrem atualmente. Em Santader,
na Espanha, uma plataforma inteligente foi desenvolvida, e, utilizando-a,
diversos aplicativos foram criados, como para apresentar aos usuários lugares
livres para estacionar, ou visualização da poluição do ar da
cidade \cite{santana2016software}. Em Amsterdã, na Holanda, um projeto permite
que os habitantes acompanhem em tempo real o consumo de energia de suas
casas \cite{kon2016}.

Esse conjunto de soluções de cidades inteligentes só são possíveis com a
utilização de três tecnologias chave: Big Data, Computação em Nuvem e Internet
das Coisas (IoT) \cite{delesposte2017}.

Big Data é uma tecnologia chave pois a massa de dados em projetos de cidades
inteligentes pode crescer de maneira absurda, e as soluções tradicionais não
são suficientes nesses casos \cite{batty2012smart}. Atualmente existem diversas
ferramentas com foco em Big Data, e, uma maneira de planejar a implantação
e utilização dessas ferramentas, é utilizando uma arquitetura já disseminada e
bem definida. Dentre essas arquiteturas, se destaca a Arquitetura Lambda, que
em sua essência lida bem com questões como tolerância a falha, escalabilidade,
e extensibilidade \cite{marz2015}.

O problema das plataformas e aplicações de cidades inteligentes atuais é que
a abordagem utilizada é normalmente muito específica e sem padronização,
não garantindo boa interoperabilidade entre os diversos projetos existentes, e
não promovendo reaproveitamento das soluções \cite{delesposte2017} criadas.

Com a finalidade de ser uma plataforma sem os defeitos citados, surge o
InterSCity, que busca promover interoperabilidade, padronização, escalabilidade
e extensibilidade \cite{delesposte2017}. A plataforma atualmente não utiliza
em seu processamento de dados tecnologias Big Data, e esta será a principal
contribuição deste trabalho. No capítulo 2 o InterSCity será detalhado,
principalmente quanto a sua arquitetura; no capítulo 3, a Arquitetura Lambda
será explicada; por fim, no capítulo 4, uma proposta de solução será feita
para o projeto InterSCity, utilizando a Arquitetura Lambda.
