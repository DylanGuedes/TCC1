\chapter[INTRODUÇÃO]{INTRODUÇÃO}

O termo \textbf{cidades inteligentes} recebe cada vez mais atenção, e trata-se
da utilização de tecnologias da informação e comunicação (TIC) para melhorar
setores como segurança, transporte, e saúde \cite{batty2012smart}.

As cidades inteligentes ganham força por abrirem soluções e mitigações
concretas para problemas graves e recorrentes das cidades atuais. Mau uso de
recursos, a burocracia, transporte de má qualidade, e falta de segurança,
podem ser resolvidos ou mitigados com soluções que utilizem
TIC \cite{batty2012smart}.

Diversas iniciativas de cidades inteligentes ocorrem atualmente. Em Santander,
na Espanha, uma plataforma inteligente foi
desenvolvida\footnote{\url{www.smartsantander.eu/}}, e, utilizando-a,
diversos aplicativos foram criados, como para apresentar aos usuários lugares
livres para estacionar
\footnote{\url{www.smartsantander.eu/wiki/index.php/Mitos/Mitos}}, ou visualização
da poluição do ar da cidade \cite{santana2016software}. Em Amsterdã, na Holanda,
um projeto permite que os habitantes acompanhem em tempo real o consumo de
energia de suas casas \cite{kon2016}.

Contudo, as plataformas e soluções atuais apresentam problemas. A abordagem
utilizada nestas soluções costuma ser muito específica e não padronizada, não
se preocupando com a interoperabilidade dentre os diversos projetos, e não
promovendo reaproveitamento das soluções existentes \cite{delesposte2017}.

Com a finalidade de ser uma plataforma sem os defeitos citados, surge o
InterSCity, que busca promover interoperabilidade, padronização,
escalabilidade, e extensibilidade \cite{delesposte2017}. O InterSCity está em
desenvolvimento, e, embora já tenha uma arquitetura bem definida, ainda não
conta com sua camada de processamento de dados.

Este trabalho tem então, como principal contribuição, o planejamento, desenho,
e implementação da camada de processamento de dados da plataforma InterSCity,
utilizando tecnologias Big Data. A arquitetura e as principais características
da plataforma serão levantadas no capítulo 2.

Big Data é uma tecnologia chave no contexto de cidades inteligentes, pois a
massa de dados em projetos deste contexto pode crescer de maneira absurda, e as
soluções tradicionais não são suficientes nesses casos \cite{batty2012smart}.
Atualmente existem diversas ferramentas com foco em Big Data, e, uma maneira de
planejar a implantação e utilização dessas ferramentas é utilizando e definindo
uma arquitetura já disseminada e testada. No capítulo 3, as arquiteturas
Lambda\footnote{\url{http://lambda-architecture.net/}} e
Kappa\footnote{\url{http://milinda.pathirage.org/kappa-architecture.com/}}
serão analisadas e detalhadas, e, ao final, no capítulo 4, será feita uma
decisão a respeito de qual arquitetura e quais tecnologias serão escolhidas
para compor a camada de processamento de dados no InterSCity.
