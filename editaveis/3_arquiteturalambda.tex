\chapter[ARQUITETURA LAMBDA]{ARQUITETURA LAMBDA}

A Arquitetura Lambda é uma abordagem para processar dados em uma plataforma
que lida com uma grande massa de dados, e surge como um caminho alternativo
a outras arquiteturas mais antigas, como a incremental com \textit{sharding}
\cite{marz2015}.

De maneira geral, a Arquitetura Lambda é composta de três camadas: a camada
\textit{batch}, que constantemente processa uma grande quantidade de dados e
demora razoávelmente em seu processamento; a camada \textit{serving},
responsável por disponibilizar os resultados do processamento; e a camada
\textit{speed}, responsável por atuar enquanto a camada \textit{batch} está
ocupada processando \cite{marz2015}. Cada uma destas camadas é implementada
utilizando algoritmos e ferramentas específicas, de modo que certas ferramentas
são mais apropriadas em certos contextos.

Um ciclo de vida típico na Arquitetura Lambda têm seu início com a (i) chegada
de novos dados, que serão então (ii) transmitidos tanto para a camada
\textit{batch} quanto para a camada \textit{speed}. A camada \textit{batch}
então (iii) anexa os novos dados, e os processa, (iv) gerando assim
\textit{batch views}, que são enviados para a camada \textit{serving}. O mesmo
dado que foi enviado para a camada \textit{batch} no passo (i), também foi
enviado para a camada \textit{speed}, (v) onde será processado com menor latência,
por só levar em conta dados recentes. O dado se encontra então processado, e
será levado em conta caso novas consultas sejam feitas, ocorrendo uma (vi) soma
entre os resultados do passo (v) e os \textit{batch views} \cite{marz2015}.

\section{BATCH LAYER}

Responsável pelo processamento de uma grande massa de dados, e tendo como ponto
fraco a alta latência, a camada \textit{batch} é responsável por gerenciar e
criar um \textit{master dataset} (um lote de informação, imutável, e que só
pode \textbf{receber} anexos) \cite{marz2015}. Os dados na camada \textit{batch}
são então imutáveis, de modo que, caso uma mudança seja necessária, o dado que
carece alteração não sofre transformações, permanecendo inalterado, mas um novo
dado com as alterações é inserido no lote \cite{marz2015}. Por fim, a camada
\textit{batch} fará o pré-processamento dos dados presentes no
\textit{master dataset}, disponibilizando-os em \textit{batch views}.

\subsection{Tecnologias}

\begin{itemize}
  \item Hadoop HDFS
  \item Apache Spark
  \item Apache Thrift
\end{itemize}

\section{SPEED LAYER}

Durante o processamento da camada \textit{batch}, novos dados podem ser
inseridos. Caso somente a \textit{batch} processasse, esses novos dados não
seriam levados em conta em novas \textit{queries}; este problema é resolvido
na Arquitetura Lambda através da camada \textit{speed}, que é capaz de
processar dados em baixa latência \cite{marz2015}.

\subsection{Tecnologias}

\begin{itemize}
  \item Spark
  \item Apache Cassandra
  \item Apache Storm
  \item Kestrel
  \item Apache Kafka
\end{itemize}

\section{SERVING LAYER}

Após os \textit{master datasets} serem criados pela camada \textit{batch}, o
projeto precisa estar pronto para responder as \textit{queries} em baixa
latêcia. Esse papel é da camada \textit{serving}, que indexa e provê interface
para os dados precomputados \cite{marz2015}.

É importante notar que os dados retornados pela camada \textit{serving} estarão
quase sempre desatualizados, pois novos dados chegarão enquanto o
\textit{master dataset} for criado.

\subsection{Tecnologias}

\begin{itemize}
    \item ElephantDB
\end{itemize}
