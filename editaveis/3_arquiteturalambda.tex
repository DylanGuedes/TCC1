\chapter[ARQUITETURA LAMBDA]{ARQUITETURA LAMBDA}

A Arquitetura Lambda é uma abordagem para processar dados em uma plataforma
que lida com uma grande massa de dados, e surge como um caminho alternativo
a outras arquiteturas mais antigas, como a incremental com \textit{sharding}
\cite{marz2015}.

De maneira geral, a Arquitetura Lambda é composta de três camadas: a camada
\textit{batch}, que constantemente processa uma grande quantidade de dados e
demora razoávelmente em seu processamento, a camada \textit{serving},
responsável por disponibilizar os resultados do processamento, e a camada
\textit{speed}, responsável por atuar enquanto a camada \textit{batch} está
ocupada processando \cite{marz2015}. Cada uma destas camadas é implementada
utilizando algoritmos e ferramentas específicas, de modo que certas ferramentas
são mais apropriadas em certos contextos.

\section{BATCH LAYER}

Responsável pelo processamento de uma grande massa de dados, a camada
\textit{batch} cria novos \textit{Master datasets} a cada ciclo de
processamento. Os dados a serem processados na camada \textit{batch} são
considerados imutáveis, de modo que, caso uma mudança seja necessária, o dado
que carece alteração não sofre transformações, permanecendo inalterado, e um
novo dado com as alterações é inserido no lote.

\subsection{Tecnologias}

\section{SERVING LAYER}

\section{SPEED LAYER}
