\begin{apendicesenv}
    \partapendices

    \chapter{PRINCÍPIOS SEGUIDOS PELO INTERSCITY}
    \label{appendix:principles}

    O InterSCity foi desenvolvido utilizando princípios de \textit{design}, e,
    assim, busca atender critérios estabelecidos. Os princípios são:

    \begin{itemize}
        \item \textbf{Modularidade através de serviços}: O InterSCity se torna mais
            modular através da criação de mais microsserviços, que buscam ter
            responsabilidades atômicas e bem definidas \cite{delesposte2017}.

        \item \textbf{Modelos e Dados Distribuídos}: O InterSCity melhora sua
            escalabilidade através da distribuição dos dados e dos modelos. Com essa
            prática, cada microsserviço pode evoluir separadamente, por contar com seu
            próprio banco de dados \cite{delesposte2017}. Contudo, esse princípio apresenta
            o ponto negativo de aumentar a complexidade.

        \item \textbf{Evolução Descentralizada}: Por conta do não-acoplamento, é
            possível que módulos do InterSCity evoluam e sofram manutenção
            independentemente, sem afetar outros microsserviços da plataforma
            \cite{delesposte2017}.

        \item \textbf{Reuso de Projetos de Código Aberto}: O InterSCity preferencia % uso software livre ou codigo aberto?
            projetos robustos já desenvolvidos, ao invés de desenvolver soluções do zero
            \cite{delesposte2017}. Contudo, essa escolha é feita com cuidado, e somente
            projetos com colaboradores e mantenedores ativos e com documentação
            apropriada são utilizadas na plataforma \cite{delesposte2017}.

        \item \textbf{Adoção de Padrões Abertos}: O InterSCity adota padrões já
            difundidos, para que seja provida maior interoperabilidade entre a plataforma
            e outros projetos \cite{delesposte2017}.

        \item \textbf{Assíncrono contra Síncrono}: O InterSCity busca prover
            serviços e atividades assíncronas sempre que possível, com a finalidade de
            evitar que eventos blocantes ocorram. Isso é atingido principalmente através
            do padrão PubSub e de estratégias baseadas em eventos \cite{delesposte2017}.

        \item \textbf{Serviços sem Estado}: Os microsserviços do InterSCity evitam,
            sempre que possível, ter um estado específico \cite{delesposte2017}. Com isso,
            os microsserviços podem responder a qualquer requisição a qualquer momento, ao
            contrário do que ocorreria caso tivessem estados específicos, pois só
            conseguiriam caso certas transições ocorressem.
    \end{itemize}
\end{apendicesenv}

